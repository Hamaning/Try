\documentclass[a4paper, 11pt]{report}
\usepackage{graphicx} % Required for inserting images
\usepackage{ragged2e}
\usepackage{xcolor}
\begin{document}
\begin{figure}[h!]
         \includegraphics[scale=1.2]{Capture d'écran 2023-11-25 193358.png}
         \end{figure}
\begin{flushleft}
   \Huge{\color{blue}\textbf{NacerBey Abderrahmane Zakaria}} \\

\Large{Etudiant à USTHB}\\
\normalsize{Matricule:232331388005\\
Section: B\\}
\vspace{2.5\baselineskip}
\Huge{\color{red}\textbf{Informations}}
\rule[0.5ex]{14cm}{1pt} 
\normalsize{Age :18\\
Nationalité: Algerien\\
Email :nacerbeyzakaria19@gmail.com}\\
\vspace{2\baselineskip}
\Huge{\color{red}\textbf{Parcours académiques}}
\rule[0.5ex]{14cm}{1pt}
\normalsize{Bac Mathélème\\
Etudiant à USTHB specialité INFO ING L1}\\
\vspace{2\baselineskip}
\Huge{\color{red}\textbf{Compétences}}
\rule[0.5ex]{14cm}{1pt}
\normalsize{programming en C\\
bureautique : Word, Excel, Powerpoint}


\end{flushleft}
\clearpage
\begin{flushleft}
    \normalsize{Alger,le 26 Novembre 2023}\\
\vspace{2\baselineskip}
Nacer Bey Abderrahmane Zakaria\\
nacerbeyzakaria19@gmail.com\\
Adresse: Bab Ezzouar,Alger,Algérie\\
\vspace{2\baselineskip}
\huge{\color{blue}\textbf{Lettre de motivation}}\\
\end{flushleft}
\normalsize{
\begin{flushright}
A l'attention de Monsieur ahmed MOHAMMED %exemple imaginaire
\\ le Directeur de l'SNI
\end{flushright}}
   Objet : Candidature pour le poste en informatique\\ 
Monsieur\\
\par
Je suis vivement intéressé par le poste développeur back-end  chez votre entreprise, tel que publié récemment. Titulaire d'un diplôme en informatique et doté d'une expérience significative dans le développement logiciel, je suis convaincu que ma passion pour l'innovation technologique et mes compétences techniques solides font de moi un candidat idéal pour contribuer à votre équipe.\\
\par
   Au cours de mon parcours professionnel chez Adhoua , j'ai pu perfectionner mes compétences en programming et \textit{problem solving}  . Mon approche proactive pour résoudre les défis techniques et ma capacité à travailler efficacement en équipe ont été des éléments clés de mes réussites antérieures. Je suis particulièrement attiré par votre entreprise , et je suis convaincu que je peux apporter une contribution significative à votre équipe.\\
\par
Je serais ravi de discuter plus en détail de ma candidature lors d'un entretien. Je vous remercie de considérer ma candidature pour le poste et me tiens à votre disposition pour toute information supplémentaire.

\vspace{2\baselineskip}
\begin{flushleft}
    Cordiales salutations,\\
    Nacer Bey Abderrahmane Zakaria
\end{flushleft}
\clearpage
\begin{flushleft}
  \Huge{\textbf{Flight List :}}\\
  \begin{table}[h!]
  \centering
      \begin{tabular}{l|c|l|r|r|c|l|}
       \hline  
           DATE & FLT & AC & DEP & ARR & STD & STA \\
    \hline
01/11/2019&	1004&	A335&	ALG&	ORY&	12:00&	14:20\\
\hline
01/11/2019&	1005&	A335&	ORY&	ALG&	15:55&	18:05\\
\hline
01/11/2019&	4062&	A335&	ALG&	DXB&	22:35&	08:05\\
\hline
01/11/2019&	4063&	A335&	DXB&	ALG&    09:50&	14:30\\
\hline
01/11/2019&	3018&	A335&	ALG&	IST&	22:30&	03:45\\
\hline
01/11/2019&	3061&	A335&	PEK&	ALG&	08:50&	14:20\\
\hline
01/11/2019&	1112&	B738&	BJA&	ORY&	10:00&	12:20\\
\hline
01/11/2019&	1113&	B738&	ORY&	BJA&	13:35&	15:50\\
\hline
01/11/2019&	6051&	B738&	BJA&	ALG&	16:50&	17:20\\
\hline
01/11/2019&	5322&	B738&	ALG&	OUA&	20:55&	0:10\\
\hline
        
      \end{tabular}

  \end{table}
\end{flushleft}
\clearpage
\begin{flushleft}
\begin{itemize}
    \item\textbf{Data quality assurance.} the aim is to provide evidence about the quality of data, this includes the following criteria:\\
    \begin{enumerate}
        \item \textit{Correctness}:\\
         \begin{itemize}
\item \textit{Accuracy}. This criterion considers the correction of data labeling and how measurement issues can affect the way that samples reflect the intended operational domain (ex. sensor accuracy).
\item \textit{Relevant}.this criterion considers that the data set is relevant to the desired behavior in the intended operational domain
         \end{itemize}
         \item The data fully reflects all possible operating conditions
         \begin{itemize}
             \item \textit{Complete}. This criterion considers the completeness of the data set over all operating scenarios.
             \item \textit{Balance.} This criterion considers the distribution of features that are included in the data set.
         \end{itemize}
    \end{enumerate}
   
    
\end{itemize}    
\end{flushleft}

\end{document}

